En todos los juegos en los que se guarde el estilo de juego, la aplicación
aconseja cómo jugar contra un oponente en concreto, recomendando al usuario aquellos
estilos que más probabilidades tengan de ganar contra ese adversario.\\

Tanto los distintos estilos como los consejos asociados a estos dependen
del juego a tratar. Por ejemplo, en juegos de estrategia como el ajedrez, se podrían
analizar las estrategias más utilizadas por un jugador y recomendar al usuario
que estudie determinadas jugadas que le puedan dar ventaja, en el caso de juegos
de mazos de cartas (e.g. Magic, Hearthstone) también sería posible recomendar algunas
configuraciones de mazos, incluso en algunos deportes se podrían recomendar
alineaciones y jugadores.\\

Debido a las diferencias en los aspectos recomendables entre distintos juegos será
necesario que cada vez que se dé soporte a un nuevo juego se especifiquen 
las posibles recomendaciones (requisitos funcionales), los datos necesarios
para llevarlas a cabo (requisitos de datos) y la forma de analizar estos datos
para proporcionar los consejos (implementación).\\

También es posible que el subsistema de consejos use funcionalidades del subsistema
de estadísticas, siempre que dicha funcionalidad esté implementada, para que las
recomendaciones sean más certeras.\\

