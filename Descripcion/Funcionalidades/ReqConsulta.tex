%TODO
\subsubsection{Requisito de datos}

\begin{itemize}
	\item \textbf{RD1.1 Juego}. Se compone de:
	\begin{itemize}
		\item Un identificador de juego.
	\end{itemize}
	
	\item \textbf{RD1.2 Partida} de un juego.  Se compone de:
	\begin{itemize}
		\item Un identificador de partida.
	\end{itemize}
	
	\item \textbf{RD1.3 Partidas del juego}. Es en realidad todos los valores de todos los atributos de una partida, un valor por cada atributo. Cada valor puede ser de cualquier tipo, por ejemplo:
	\begin{itemize}
		\item Valor númerico.
		\item Una cadena de caracteres.
		\item Una letra.
		\item Una imagen.
		\item Un vídeo.
	\end{itemize}
	
	\item \textbf{RD1.4 Atributo} de una partida. Se encuentra añadido en la parte previa a la gestión de la base de datos.
	
	\item \textbf{RD1.5 Valor} de un atributo. Puede ser de cualquier tipo, por ejemplo:
	\begin{itemize}
		\item Valor númerico.
		\item Una cadena de caracteres.
		\item Una letra.
		\item Una imagen.
		\item Un vídeo.
	\end{itemize}
	
	\item \textbf{RD1.6 Atributos} de una partida. Es un conjunto de varios atributos añadidos en la parte previa a la gestión de la base de datos.
	
	\item \textbf{RD1.7 Valores de un mismo atributo}. Puede ser de cualquier tipo, aunque todos del mismo tipo, por ejemplo:
	\begin{itemize}
		\item Valor númerico.
		\item Una cadena de caracteres.
		\item Una letra.
		\item Una imagen.
		\item Un vídeo.
	\end{itemize}
	
	\item \textbf{RD1.8 Lista de condiciones}, de las cuales necesitan un valor de atributo, que son:
	\begin{itemize}
		\item 'sea menor que'.
		\item 'sea menor o igual que'.
		\item 'sea igual que'.
		\item 'sea mayor o igual que'.
		\item 'sea mayor que'.
		\item 'sea' (comparación entre cadena de caracteres).
		\item 'no sea' (comparación entre cadena de caracteres).
	\end{itemize}
	
	\item \textbf{RD1.9 Valores de varios atributos}. Pueden ser de cualquier tipo, por ejemplo:
	\begin{itemize}
		\item Valor númerico.
		\item Una cadena de caracteres.
		\item Una letra.
		\item Una imagen.
		\item Un vídeo.
	\end{itemize}
\end{itemize}




\subsubsection{Requisitos funcionales}
\begin{itemize}
	\item \textbf{RF1.1 Consultar partidas}: Consultar todas las partidas de un juego.
	
	Requisitos de entrada:
	\begin{itemize}
		\item RD1.1
	\end{itemize}
	Manejamiento de datos: Se lee el identificador del juego.
	\begin{itemize}
		\item RD1.1
	\end{itemize}
	Salida:
	\begin{itemize}
		\item RD1.3
	\end{itemize}
	
	
	\item \textbf{RF1.2 Restricción por condición}: Sobre un atributo, exigir que cumpla una condición dado un valor. Para facilitar esta función, sólo se puede exigir que se cumpla una condición sobre una lista de condiciones dada. Por ejemplo, Por ejemplo, si existiese un atributo en un juego que sólo tiene como valores {Ganado, Empatado, Perdido}, poder restringir a aquellas partidas donde dicho atributo sea 'Ganado'; o para el atributo 'Puntuación', poder elegir las partidas donde 'Se haya ganado por más de 10 puntos', donde la condición sobre el atributo 'Puntuación' sería 'es mayor que' y el valor pedido sería '10'.
	
	Requisitos de entrada:
	\begin{itemize}
		\item RD1.1
		\item RD1.4
		\item RD1.8
		\item RD1.5
	\end{itemize}
	Manejamiento de datos: Se lee el identificador del juego, el atributo a restringir, la lista de condiciones y el valor para la condición.
	\begin{itemize}
		\item RD1.1
		\item RD1.4
		\item RD1.8
		\item RD1.5
	\end{itemize}
	Salida:
	\begin{itemize}
		\item RD1.7
	\end{itemize}
	
	
	\item \textbf{RF1.3 Unión de restricciones}: Poder hacer la función 'OR' entre restricciones en una consulta. Por ejemplo, "todos los partidos en los que ganó o empató el Unicaja" o "partidos en los que ganó el Unicaja o la diferencia en la puntuación fue menos de 10".
	
	Requisitos de entrada:
	\begin{itemize}
		\item RD1.1
		\item RD1.4
		\item RD1.8
		\item RD1.9
	\end{itemize}
	Manejamiento de datos: Se lee el identificador del juego, los atributos a realizar la unión de restricciones, la lista de condiciones donde se selecciona una condición por atributo y los valores para cada condición.
	\begin{itemize}
		\item RD1.1
		\item RD1.6
		\item RD1.8
		\item RD1.9
	\end{itemize}
	Salida:
	\begin{itemize}
		\item RD1.9
	\end{itemize}
	
	
	\item \textbf{RF1.4 Intersección de restricciones}: Poder hacer la función 'AND' entre restricciones en una consulta. Por ejemplo, "todos los partidos en los que ganó y empató el Unicaja" o "partidos en los que ganó el Unicaja y la diferencia en la puntuación fue menos de 10".
	
	Requisitos de entrada:
	\begin{itemize}
		\item RD1.1
		\item RD1.4
		\item RD1.8
		\item RD1.9
	\end{itemize}
	Manejamiento de datos: Se lee el identificador del juego, los atributos a realizar la intersección de restricciones, la lista de condiciones donde se selecciona una condición por atributo y los valores para cada condición.
	\begin{itemize}
		\item RD1.1
		\item RD1.6
		\item RD1.8
		\item RD1.9
	\end{itemize}
	Salida:
	\begin{itemize}
		\item RD1.9
	\end{itemize}
\end{itemize}




\subsubsection{Restricciones semánticas}

\textbf{RS1.1}. Para RF1.2, el valor de RD1.5 que acompaña a los 'es menor que', 'es menor o igual que', 'es igual que', 'es mayor o igual que', 'es mayor que' de la lista de condiciones RD1.8 ha de ser numérico.