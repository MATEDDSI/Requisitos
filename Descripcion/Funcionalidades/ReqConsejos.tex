\subsubsection{Requisitos de datos}

	\begin{itemize}
		\item \textbf{RD1: Jugador de pokémon} contra el que diseñar un equipo. Se
			requiere la identificación del jugador.

		\item \textbf{RD2: Debilidades y fortalezas de tipos pokémon,} para
			cada tipo se require del multiplicador de ataque contra los
			demás tipos.\\

		\item \textbf{RD3: Información de ataques pokémon,} para cada ataque
			se requiere:
			\begin{itemize}
				\item Tipo del ataque
				\item Potencia del ataque
				\item Precisión del ataque
				\item Si es físico, especial o de estado
				\item Puntos de poder
			\end{itemize}

		\item \textbf{RD4: Estadísticas base de los pokemon,} para cada pokémon
			se requiere:
			\begin{itemize}
				\item Puntos de salud
				\item Puntos de ataque
				\item Puntos de defensa
				\item Puntos de ataque especial
				\item Puntos de defensa especial
				\item Puntos de velocidad
			\end{itemize}

		\item \textbf{RD5: Equipos pokémon usados por un jugador} en las partidas
			registradas. Se compone de:
			\begin{itemize}
				\item Una lista de como máximo seis pokémon por partida.
				\item Los atributos relacionados con cada pokemon
					(opcionales), estos son:
					\begin{itemize}
						\item Id del pokémon
						\item Nombre de los ataques
						\item Objeto asignado
						\item Naturaleza
						\item Distibución de IVs
						\item Distribución de EVs
					\end{itemize}
			\end{itemize}

		\item \textbf{RD6: Equipo pokémon} diseñado para vencer a un oponente.
			Consta de:
			\begin{itemize}
				\item Una lista de como máximo seis pokémon.
				\item Los atributos relacionados con cada pokemon
					(opcionales), estos son:
					\begin{itemize}
						\item Id del pokémon
						\item Nombre de los ataques
						\item Objeto asignado
						\item Naturaleza
						\item Distibución de IVs
						\item Distribución de EVs
					\end{itemize}
			\end{itemize}
	\end{itemize}


\subsubsection{Requisitos funcionales}

	\begin{itemize}
		\item \textbf{RF1: Recomendar equipos pokémon:}\\

		Requisitos de entrada:
		\begin{itemize}
			\item RD1
			\item RD2
			\item RD3
			\item RD4
		\end{itemize}

		Manejamiento de datos:
		\begin{itemize}
			\item RD5
		\end{itemize}

		Salida:
		\begin{itemize}
			\item RD6
		\end{itemize}
	\end{itemize}


\subsubsection{Restricciones semánticas}
