\subsubsection{Requisitos de datos}

	\begin{itemize}
		\item \textbf{RD2.1: Juego} sobre el que se va a aconsejar.
			Se require un identificador del juego.

		\item \textbf{RD2.2: Atributos} para los que se quiere consejo.
			Requiere un identificador de cada atributo.

		\item \textbf{RD2.3: Oponente} contra el que optimizar el atributo.
			Requiere un identificador del oponente.

		\item \textbf{RD2.4: Partidas del juego} almacenadas.
			Se requerirán los campos que se considere que tienen información
			relevante para el consejo.

		\item \textbf{RD2.5: Valor de los atributos} optimizado contra el oponente.

		\item \textbf{RD2.6: Juego} sobre el que se va a aconsejar.
			Se require un identificador del juego.

		\item \textbf{RD2.7: Atributos} para los que se quiere consejo.
			Requiere un identificador de cada atributo.

		\item \textbf{RD2.8: Valor de los atributos} optimizado contra todos los
			oponentes almacenados.

		\item \textbf{RD2.9: Juego} sobre el que se va a aconsejar.
			Se require un identificador del juego.

		\item \textbf{RD2.10: Atributos} para los que se quiere consejo.
			Requiere un identificador de cada atributo.

		\item \textbf{RD2.11: Valor de los atributos} de un supuesto oponente.
			Consta de una lista de valores junto con el identificador del
			atributo al que esta asocido cada uno.

		\item \textbf{RD2.12: Valor de los atributos} optimizado contra un oponente
			con los atributos especificados.

		\item \textbf{RD2.13: Juego} sobre el que se va a aconsejar.
			Se require un identificador del juego.

		\item \textbf{RD2.14: Atributos} para los que se quiere consejo.
			Requiere un identificador de cada atributo.

		\item \textbf{RD2.15: Jugador} del que se obtendrán atributos semejantes.
			Requiere un identificador del jugador.

		\item \textbf{RD2.16: Valor de los atributos} adaptado para que se parezca
		 	al del jugador especificado.
	\end{itemize}


\subsubsection{Requisitos funcionales}

	\begin{itemize}
		\item \textbf{RF2.1: Recomendar atributos contra un oponente:}
			El usuario selecciona uno o varios atributos del juego y un
			oponente y el sistema le proporciona valores de los
			atributos que probablemente sean beneficiosos contra ese oponente.\\

			Requisitos de entrada: El usuario primero selecciona un juego y
			después se le presentan los atributos de este y otros
			jugadores registrados, para que escoja los atributos sobre
			los que quiere ser aconsejado y el oponenete.
			\begin{itemize}
				\item RD2.1 %juego
				\item RD2.2 %atributos
				\item RD2.3 %oponente
			\end{itemize}
			Manejo de datos: El sistema lee varias partidas registradas
			en la base de datos, en las que el oponente sea relevante,
			y hace inferencia sobre ellas para aconsejar buenos atributos.
			\begin{itemize}
				\item RD2.4 %partidas relevantes
			\end{itemize}
			Salida: Se le presentan al usuario los valores optimizados
			para cada atributo escogido.
			\begin{itemize}
				\item RD2.5 %valor atributo
			\end{itemize}

		\item \textbf{RF2.2: Recomendar atributos en general:}
			El usuario selecciona uno o varios atributos del juego
			y el sistema le proporciona valores de los atributos
			que probablemente sean beneficiosos en la mayoría
			de partidas que juegue.\\

			Requisitos de entrada: El usuario primero selecciona un juego y
			después se le presentan los atributos para que escoja sobre
			los que quiere ser aconsejado.
			\begin{itemize}
				\item RD2.6 %juego
				\item RD2.7 %atributos
			\end{itemize}
			Manejo de datos: El sistema lee varias partidas registradas
			en la base de datos y hace inferencia sobre ellas para
			aconsejar buenos atributos.
			\begin{itemize}
				\item RD2.4 %partidas relevantes
			\end{itemize}
			Salida: Se le presentan al usuario los valores optimizados
			para cada atributo escogido.
			\begin{itemize}
				\item RD2.8 %valor atributo
			\end{itemize}

		\item \textbf{RF2.3: Recomendar atributos con contexto:}
			El usuario selecciona uno o varios atributos del juego y
			el valor de uno o varios atributos del adversrio y el sistema
			proporciona un valor para los atributos seleccionados que
			probablemente sean beneficiosos contra un adversario con
			los atributos especificados.\\

			Requisitos de entrada: El usuario primero selecciona un juego y
			después se le presentan los atributos para que escoja sobre
			los que quiere ser aconsejado y los valores de los de un
			oponente hipotético.
			\begin{itemize}
				\item RD2.9 %juego
				\item RD2.10 %atributo
				\item RD2.11 %atributos oponente
			\end{itemize}
			Manejo de datos: El sistema lee varias partidas registradas
			en la base de datos, en las que los valores de los atributos
			del oponente sean relevantes, y hace inferencia sobre ellas para
			aconsejar buenos atributos.
			\begin{itemize}
				\item RD2.4 %partidas relevantes
			\end{itemize}
			Salida: Se le presentan al usuario los valores optimizados
			para cada atributo escogido.
			\begin{itemize}
				\item RD2.12 %valor atributo
			\end{itemize}

	\item \textbf{RF2.4: Recomendar un estilo de juego parecido al
		de un jugador:}
		El usuario selecciona uno o varios atributos y el sistema
		proporciona valores que se asemejan a los que suele tener
		un jugador concreto.\\

		Requisitos de entrada:
		\begin{itemize}
			\item RD2.13 %juego
			\item RD2.14 %atriibutos
			\item RD2.15 %jugador
		\end{itemize}
		Manejo de datos:
		\begin{itemize}
			\item RD2.4
		\end{itemize}
		Salida:
		\begin{itemize}
			\item RD2.16 %atributos salida
		\end{itemize}

	\end{itemize}


\subsubsection{Restricciones semánticas}
	%\begin{itemize}
	%	\item \textbf{RS2.1 Los atributos deben pertenecer al juego} 
	%		\begin{itemize}
	%			\item RF2.1
	%			\item RD2.1
	%			\item RD2.2
	%		\end{itemize}
	%\end{itemize}
