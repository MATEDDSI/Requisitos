%TODO
\subsubsection{Requisito de datos}
\begin{itemize}
	\item \textbf{RD4.1: Atributos de entrada a una gráfica 2D}, proporcionados por el usuario, se componen de:
	\begin{itemize}
		\item Atributo de la partida sobre la X
		\item Atributo de la partida sobre la Y
	\end{itemize}
	
	\item \textbf{RD4.2: Atributos de una partida}, se componen de:
	\begin{itemize}
		\item Nombre
		\item Valor del atributo
	\end{itemize}
	
	\item \textbf{RD4.3: Gráfica 2D realizada}, se compone de:
	\begin{itemize}
		\item Imagen con la gráfica
	\end{itemize}
	
	\item \textbf{RD4.4: Atributos de entrada a una gráfica 3D}, se componen de:
	\begin{itemize}
			\item Atributo de la partida sobre la X
			\item Atributo de la partida sobre la Y
			\item Atributo de la partida sobre la Z
	\end{itemize}
	
	
	\item \textbf{RD4.5: Gráfica 3D}, se compone de:
	\begin{itemize}
			\item Imagen con la gráfica
	\end{itemize}
	
	\item \textbf{RD4.6: Atributos de entrada a una gráfica de columnas agrupadas}, se componen de:
	\begin{itemize}
		\item Atributo de la partida sobre la X
		\item Atributo de primer tipo de columna
		\item Atributo del segundo tipo de columna
	\end{itemize}
	
	\item \textbf{RD4.7: Gráfica de columnas agrupadas}, se compone de:
	\begin{itemize}
		\item Imagen con la gráfica
	\end{itemize}
	
	\item \textbf{RD4.8: Atributos de entrada a una gráfica circular}, se componen de:
	\begin{itemize}
		\item Atributo sobre el que se hace el círculo
	\end{itemize}
	
	\item \textbf{RD4.9: Gráfica circular}, se compone de:
	\begin{itemize}
			\item Imagen con la gráfica
	\end{itemize}
	
	\item \textbf{RD4.10: Atributo y función elemental}, se compone de:
	\begin{itemize}
		\item Atributo sobre el que se hace la función
		\item Función, a elegir, entre media, varianza, suma, contar, máximo o mínimo
	\end{itemize}
	
	\item \textbf{RD4.11: Resultado de la función}, se compone de:
	\begin{itemize}
		\item Resultado de la operación sobre el atributo
	\end{itemize}
	
	\item \textbf{RD4.12: Petición de conseguir las últimas gráficas}, se compone de:
	\begin{itemize}
		\item Activación de petición de la gráfica (pulsación de un botón)
	\end{itemize}
	
	\item \textbf{RD4.13: Gráficas anteriores}, se compone de:
	\begin{itemize}
		\item Imágenes de gráficas
	\end{itemize}
	
	\item \textbf{RD4.14: Gráficas anteriores como salida}, se compone de:
	\begin{itemize}
		\item Imágenes de gráficas
	\end{itemize}
	
	
	
	
	
\end{itemize}
\subsubsection{Requisitos funcionales}
\begin{itemize}
	\item \textbf{RF4.1 Realizar gráfica 2D de unos atributos}: El usuario selecciona dos atributos, de los cuales se realiza una gráfica 2D
	
	Requisitos de entrada: Elegir qué dos atributos se escogen
	\begin{itemize}
		\item RD4.1
	\end{itemize}
	Manejo de datos: Se leen los dos atributos seleccionados  y se inserta la gráfica en la base de datos
	\begin{itemize}
		\item RD4.2
	\end{itemize}
	Salida: Gráfica 2D
	\begin{itemize}
		\item RD4.3
	\end{itemize}
	
	\item \textbf{RF4.2 Realizar gráfica 3D de unos atributos}: El usuario selecciona tres atributos, de los cuales se realiza una gráfica 3D
	
	Requisitos de entrada: Elegir qué tres atributos se escogen
	\begin{itemize}
		\item RD4.4
	\end{itemize}
	Manejo de datos: Se leen los tres atributos seleccionados  y se inserta la gráfica en la base de datos
	\begin{itemize}
		\item RD4.2
	\end{itemize}
	Salida: Gráfica 3D
	\begin{itemize}
		\item RD4.5
	\end{itemize}
	
	\item \textbf{RF4.3 Realizar gráfica de columnas agrupadas de unos atributos}: El usuario selecciona dos atributos para las columnas y un tercero para el eje de abscisas, de los cuales se realiza la gráfica
	
	Requisitos de entrada: Elegir qué atributos se escogen
	\begin{itemize}
		\item RD4.6
	\end{itemize}
	Manejo de datos: Se leen los atributos seleccionados y se inserta la gráfica en la base de datos.
	\begin{itemize}
		\item RD4.2
	\end{itemize}
	Salida: Gráfica
	\begin{itemize}
		\item RD4.7
	\end{itemize}
	
	\item \textbf{RF4.4 Realizar gráfica circular de unos atributos}: El usuario selecciona un atributo, del cual se hace una gráfica porcentual en forma de círculo
	
	Requisitos de entrada: Elegir qué atributos se escoge.
	\begin{itemize}
		\item RD4.8
	\end{itemize}
	Manejo de datos: Se lee el atributo seleccionado y se inserta la gráfica en la base de datos.
	\begin{itemize}
		\item RD4.2
	\end{itemize}
	Salida: Gráfica.
	\begin{itemize}
		\item RD4.9
	\end{itemize}
	
	\item \textbf{RF4.5 Realizar una función elemental sobre un atributo concreto}: El usuario selecciona una función elemental (detalladas en la entrada de este requisito) y un atributo sobre el que operar, y se devuelve el resultado de la operación.
	
	Requisitos de entrada: El atributo y la función, a elegir entre media, varianza, suma, contar, máximo o mínimo.
	\begin{itemize}
		\item RD4.10
	\end{itemize}
	Manejo de datos: Se lee el atributo seleccionado y se opera.
	\begin{itemize}
		\item RD4.2
	\end{itemize}
	Salida: Resultado de la función.
	\begin{itemize}
		\item RD4.11
	\end{itemize}
	
	\item \textbf{RF4.6 Conseguir las últimas gráficas pedidas}: El usuario le da a un botón para ver las últimas gráficas que ha pedido.
	
	Requisitos de entrada: La activación de un botón.
	\begin{itemize}
		\item RD4.12
	\end{itemize}
	Manejo de datos: Se cogen las gráficas realizadas en función de su fecha.
	\begin{itemize}
		\item RD4.13
	\end{itemize}
	Salida: Gráficas, ordenadas de más a menos recientes.
	\begin{itemize}
		\item RD4.14
	\end{itemize}
	
\end{itemize}

\subsubsection{Restricciones semánticas}

\begin{itemize}
	\item \textbf{RS4.1 El atributo sobre la Y de RD4.1 sobre el requisito funcional de hacer una gráfica 2D RF4.1 debe ser de tipo numérico.}
	
	\item \textbf{RS4.2 El atributo sobre la Z de RD4.4 sobre el requisito funcional de hacer una gráfica 3D RF4.2 debe ser de tipo numérico.}
	
	\item \textbf{RS4.3 El atributo de los tipos de columnas en RD4.6 sobre el requisito funcional de hacer una gráfica de columnas agrupadas en RF4.3 deben ser de tipo numérico.}
	
	\item \textbf{RS4.4 El atributo de entrada a una gráfica circular en RD4.8 sobre el requisito funcional RF4.4 debe ser de tipo numérico y positivos.}
	
	\item \textbf{RS4.5 Los atributos de entrada a una función elemental en RD4.10 sobre el requisito funcional RF4.5 deben ser de tipo numérico.}
	
	
	
\end{itemize}