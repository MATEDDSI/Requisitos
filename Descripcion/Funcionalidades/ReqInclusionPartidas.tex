%TODO
\subsubsection{Requisito de datos}
\begin{itemize}
	\item \textbf{RD3.1: Atributos Principales}, proporcionados por el usuario, se componen de:
	\begin{itemize}
		\item Puntuación de la partida.
		\item Nombre del juego
	\end{itemize}
	
	\item \textbf{RD3.2: Atributos Secundarios}, se componen de:
	\begin{itemize}
		\item Nombre
		\item Valor del atributo
	\end{itemize}
	La lista de atributos secundarios la definirań los administradores del sistema.
\end{itemize}



\subsubsection{Requisitos funcionales}

\begin{itemize}
	\item \textbf{RF3.1 Selección del juego}: El usuario selecciona uno de entre los juegos que recoge el sistema.
	\begin{itemize}
		\item RD3.1
	\end{itemize}
	Salida: En pantalla se muestran los juegos posibles.
	
	\item \textbf{RF3.2 Inclusión de una partida}: El usuario inserta el atributo principal y crea un nuevo registro de partida.
	\begin{itemize}
		\item RD3.1
	\end{itemize}
	Requisitos de entrada: Una vez escogido el juego se inserta la puntuación de la partida.
	
	Manejo de datos: Se leen el atributo proporcionado y se el registro en la base de datos
	
	Salida: Mensaje de conformidad.
	
	
	\item \textbf{RF3.3 Modificación de un registro}: Una vez creado, el usuario podrá añadir atributos opcionales al registro.
	
	Requisitos de entrada: Lista de los atributos a insertar
	\begin{itemize}
		\item RF3.2
	\end{itemize}
	Manejo de datos: Se leen los atributos seleccionados y se inserta la gráfica en la base de datos.
	\begin{itemize}
		\item RD3.2
	\end{itemize}
	Salida: A pantalla se muestra una interfaz en la que se representa el registro completo.
	
\end{itemize}	
	
	\subsubsection{Restricciones semánticas}
