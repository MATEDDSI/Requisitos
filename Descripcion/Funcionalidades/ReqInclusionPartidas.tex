%TODO
\subsubsection{Requisito de datos}

\begin{itemize}
	\item \textbf{RD3.1: Atributos Principales}, proporcionados por el usuario, se componen de:
	\begin{itemize}
		\item Puntuación de la partida.
		\item Nombre del juego
	\end{itemize}

	\item \textbf{RD3.2: Atributos Secundarios}, se componen de:
	\begin{itemize}
		\item Nombre
		\item Valor del atributo
	\end{itemize}
	La lista de atributos secundarios la definirań los administradores del sistema.

	\item \textbf{RD3.3 Identificador de partida} Asignado por el sistema.

	\item \textbf{RD3.4 Comentarios de otros jugadores} Se permitirá asociar un comentario a la partida de otro jugador.

	\item \textbf{RD3.5 Registro de partida} Se compone de:
	\begin{itemize}
		\item Atributos principales (obligatorio)
		\item Atributos secundarios (opcionales)
		\item Identificador de partida (obligatorio)
		\item Comentario de otros jugadores (opcional)
	\end{itemize}

\end{itemize}



\subsubsection{Requisitos funcionales}

\begin{itemize}

	\item \textbf{RF3.1 Inclusión de una partida}: El usuario inserta una pareja que consta de nombre de juego y puntuación, creando un nuevo registro de partida.

	Requisitos de entrada: Nombre del juego y puntuación
	\begin{itemize}
		\item RD3.1
	\end{itemize}

	Manejo de datos: Se leen los atributos, en caso de que el nombre del juego sea aceptado, se crea el registro.
	\begin{itemize}
		\item RD3.5
	\end{itemize}

	Salida: Identificador de la partida.
	\begin{itemize}
		\item RD3.3
	\end{itemize}

	\item \textbf{RF3.2 Modificación de un registro}: Una vez creado, el usuario podrá añadir atributos opcionales al registro.

	Requisitos de entrada: Lista de los atributos a insertar e identifificador de la partida.\\
	\begin{itemize}
		\item RD3.3
		\item RD3.5
	\end{itemize}

	Manejo de datos: Se leen los atributos seleccionados y se insertan en el registro de la partida en cuestión. \\
	\begin{itemize}
		\item RD3.2
	\end{itemize}

	Salida: A pantalla se muestra una interfaz en la que se representa el registro completo.
	\begin{itemize}
		\item RD3.5
	\end{itemize}

	\item \textbf{RF3.3 Eliminación de un registro} Se permitirá eliminar un resgitro en concreto si poseemos su identificador.

	Requisito de entrada: Identificador de la partida.
	\begin{itemize}
		\item RD3.3
	\end{itemize}

	Manejo de datos: Se busca el registro y se elimina de la base de datos. \\
	\begin{itemize}
		\item RD5.5
	\end{itemize}

	Salida: Mensaje de conformidad. \\

	\item \textbf{RF3.4 Comentar la partida de otro jugador}

	Requisito de entrada: String con el comentario.
	\begin{itemize}
		\item RD3.4
		\item RD3.3
	\end{itemize}

	Manejo de datos: Se asocia un comentario a la partida de otro jugador
	\begin{itemize}
		\item RD3.5
	\end{itemize}

	Salida: Mensaje de conformidad \\

\end{itemize}

	\subsubsection{Restricciones semánticas}

	\begin{itemize}
		\item \textbf{RS3.1 La puntuación de una partida es una pareja (a,b) donde a y b son naturales.}
			\begin{itemize}
				\item RD3.1
				\item RF3.1
			\end{itemize}
		\item \textbf{RS3.2 El identificador de una partida será un entero.}
			\begin{itemize}
				\item RD3.3
				\item RF3.1
				\item RF3.3
			\end{itemize}

		\item \textbf{No se permitirá asociar un string vacío a la sección de comentarios de una partida.}
		\begin{itemize}
			\item RD3.4
			\item RF3.4
		\end{itemize}
	\end{itemize}
