\documentclass[a4paper, 11pt]{article}

\usepackage[spanish]{babel}
\usepackage[utf8]{inputenc}
\usepackage[vmargin=2cm,hmargin=2cm]{geometry}
\usepackage{enumerate}
\usepackage{dsfont}
\usepackage{graphicx}

\title{\Huge \textbf{Descripción del problema\\DDSI}}

\author{Antonio Checa Molina \\ Iñaki Madinabeitia \\ Bruno Santidrián \\ Darío Sierra}

\date{\today}


\begin{document}

\maketitle
\tableofcontents

\newpage
\section{Gestión de un registro de partidas en juegos generales}

El problema a resolver es mantener un registro rápido y funcional de partidas para cualquier tipo de juegos, deportes o competiciones con el fin de facilitar análisis y entrenamiento de jugadores. Hay que desarrollar un sistema de información de organización de las partidas y de los juegos.

Al principio, un grupo de gestores necesita proporcionar información de un juego y de sus partidas: aquello que se guarda, como la puntuación, los equipos, los jugadores de los equipos o un vídeo del partido. Una vez guardados estos atributos, el usuario podrá acceder al registro de partidas de un juego concreto e incluir aquellas de las que tenga datos. 

Por ejemplo, si se añade el juego baloncesto junto a un conjunto de atributos como \{Puntuación, Equipos, Puntuación de cada parcial, vídeos\}, cada partida contendrá esta información y los usuarios podrán acceder a esta. En esta práctica nos restringiremos a un pequeño número de juegos, realizando el diseño de base de datos para cada uno.

\subsection{Consulta}
\subsubsection{Diagramas de flujo de datos:}

Refinamos el proceso Consulta en sus cuatro requisitos funcionales  y realizando la descomposición del almacén en Partidas y Atributos mediante una primitiva descendente de descomposición en procesos sin conexiones.
 
\begin{figure}[h!]
\centering
\includegraphics[width=0.7\linewidth]{../Diagramas/pdf/RefinamientoConsulta.pdf}
\caption{Flujo de datos del subsistema Consulta.}

\label{fig:RefinamientoConsulta}
\end{figure}

\subsubsection{Esquema Externo:}

\begin{figure}[h!]
	\centering
	\includegraphics[width=0.7\linewidth]{../Diagramas/pdf/EsquemaExternoConsulta.pdf}
	\caption{Esquema externo de Consulta.}
	
	\label{fig:EsquemaExtConsulta}
\end{figure}


\subsection{Consejos}
\subsubsection{Diagramas de flujo de datos:}

Refinamos el proceso Consejos del diagrama armazón mediante
una primitiva descendente de descomposición en procesos sin conexiones.\\
Simultaneamente realizamos una descomposición de almacen sobre el
almacén MateDB del armazón.\\
Por último conectamos las descomposiciones resultantes mediante dos
primitivas ascendentes de generación de flujo.\\ 
 
\begin{figure}[H]
\centering
\includegraphics[width=0.7\linewidth]{../Diagramas/pdf/RefinamientoConsejos.pdf}
\caption[Flujo de datos del subsistema Consejos]{}
\caption{}
\label{fig:RefinamientoConsejos}
\end{figure}


\subsection{Inclusión de partidas}
La inclusión de las partidas necesita que el usuario incluya aquellos atributos vitales en una partida, pudiendo o no incluir los opcionales. Necesitaría insertarlos por la interfaz gráfica.


\subsection{Estadísticas}
\subsubsection{Diagramas de flujo de datos:}

Refinamos el proceso de realizar gráfica en los cuatro requisitos funcionales junto al almacén de datos, que separamos en los almacenes Partidas y Estadísticas.
 
 
 \begin{figure}[h!]
 	\centering
 	\includegraphics[width=0.7\linewidth]{../Diagramas/pdf/DiagramaEstadistica1.pdf}
 	\caption{Subsistema de estadísticas, nivel 1}
 	\label{fig:RefinamientoEstadisticas1}
 \end{figure}
 
 
 \begin{figure}[h!]
 	\centering
 	\includegraphics[width=0.7\linewidth]{../Diagramas/pdf/DiagramaEstadistica2.pdf}
 	\caption{Subsistema de estadísticas, nivel 2}
 	\label{fig:RefinamientoEstadisticas2}
 \end{figure}
 
 
\begin{figure}[h!]
\centering
\includegraphics[width=0.7\linewidth]{../Diagramas/pdf/RefinamientoEstadisticas.pdf}
\caption{Refinamiento del diagrama de estadísticas}
\label{fig:RefinamientoEstadisticas}
\end{figure}

	
\end{document}
