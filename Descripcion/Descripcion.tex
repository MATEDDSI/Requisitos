\documentclass[a4paper, 11pt]{article}

\usepackage[spanish]{babel}
\usepackage[utf8]{inputenc}
\usepackage[vmargin=2cm,hmargin=2cm]{geometry}
\usepackage{enumerate}
\usepackage{dsfont}
\usepackage{graphicx}

\title{\Huge \textbf{Descripción del problema\\DDSI}}

\author{Antonio Checa Molina \\ Iñaki Madinabeitia \\ Bruno Santidrián \\ Darío Sierra}

\date{\today}


\begin{document}

\maketitle
\tableofcontents

\newpage
\section{Gestión de un registro de partidas en juegos generales}

El problema a resolver es mantener un registro rápido y funcional de partidas para cualquier tipo de juegos, deportes o competiciones con el fin de facilitar análisis y entrenamiento de jugadores. Hay que desarrollar un sistema de información de organización de las partidas y de los juegos.

Al principio, un grupo de gestores necesita proporcionar información de un juego y de sus partidas: aquello que se guarda, como la puntuación, los equipos, los jugadores de los equipos o un vídeo del partido. Una vez guardados estos atributos, el usuario podrá acceder al registro de partidas de un juego concreto e incluir aquellas de las que tenga datos.

Por ejemplo, si se añade el juego baloncesto junto a un conjunto de atributos como \{Puntuación, Equipos, Puntuación de cada parcial, vídeos\}, cada partida contendrá esta información y los usuarios podrán acceder a esta. En esta práctica nos restringiremos a un pequeño número de juegos, realizando el diseño de base de datos para cada uno.

\subsection{Consulta (Iñaki Madinabeitia)}
\subsubsection{Diagramas de flujo de datos:}

Refinamos el proceso Consulta en sus cuatro requisitos funcionales  y realizando la descomposición del almacén en Partidas y Atributos mediante una primitiva descendente de descomposición en procesos sin conexiones.
 
\begin{figure}[h!]
\centering
\includegraphics[width=0.7\linewidth]{../Diagramas/pdf/RefinamientoConsulta.pdf}
\caption{Flujo de datos del subsistema Consulta.}

\label{fig:RefinamientoConsulta}
\end{figure}

\subsubsection{Esquema Externo:}

\begin{figure}[h!]
	\centering
	\includegraphics[width=0.7\linewidth]{../Diagramas/pdf/EsquemaExternoConsulta.pdf}
	\caption{Esquema externo de Consulta.}
	
	\label{fig:EsquemaExtConsulta}
\end{figure}


\subsection{Consejos (Bruno Santidrián)}
\subsubsection{Diagramas de flujo de datos:}

Refinamos el proceso Consejos del diagrama armazón mediante
una primitiva descendente de descomposición en procesos sin conexiones.\\
Simultaneamente realizamos una descomposición de almacen sobre el
almacén MateDB del armazón.\\
Por último conectamos las descomposiciones resultantes mediante dos
primitivas ascendentes de generación de flujo.\\ 
 
\begin{figure}[H]
\centering
\includegraphics[width=0.7\linewidth]{../Diagramas/pdf/RefinamientoConsejos.pdf}
\caption[Flujo de datos del subsistema Consejos]{}
\caption{}
\label{fig:RefinamientoConsejos}
\end{figure}


\subsection{Inclusión de partidas (Darío Sierra)}
La inclusión de las partidas necesita que el usuario incluya aquellos atributos vitales en una partida, pudiendo o no incluir los opcionales. Necesitaría insertarlos por la interfaz gráfica.


\subsection{Estadísticas (Antonio Checa)}
\subsubsection{Diagramas de flujo de datos:}

Refinamos el proceso de realizar gráfica en los cuatro requisitos funcionales junto al almacén de datos, que separamos en los almacenes Partidas y Estadísticas.
 
 
 \begin{figure}[h!]
 	\centering
 	\includegraphics[width=0.7\linewidth]{../Diagramas/pdf/DiagramaEstadistica1.pdf}
 	\caption{Subsistema de estadísticas, nivel 1}
 	\label{fig:RefinamientoEstadisticas1}
 \end{figure}
 
 
 \begin{figure}[h!]
 	\centering
 	\includegraphics[width=0.7\linewidth]{../Diagramas/pdf/DiagramaEstadistica2.pdf}
 	\caption{Subsistema de estadísticas, nivel 2}
 	\label{fig:RefinamientoEstadisticas2}
 \end{figure}
 
 
\begin{figure}[h!]
\centering
\includegraphics[width=0.7\linewidth]{../Diagramas/pdf/RefinamientoEstadisticas.pdf}
\caption{Refinamiento del diagrama de estadísticas}
\label{fig:RefinamientoEstadisticas}
\end{figure}


\newpage
\section{Análisis de requisitos}

\subsection{Primer subsistema, consulta}
\subsubsection{Requisito de datos}

\begin{itemize}
	\item \textbf{RD1.1 Juego}. Se compone de:
	\begin{itemize}
		\item Un identificador de juego.
	\end{itemize}
	
	\item \textbf{RD1.2 Partida} de un juego.  Se compone de:
	\begin{itemize}
		\item Un identificador de partida.
	\end{itemize}
	
	\item \textbf{RD1.3 Partidas del juego}. Es en realidad todos los valores de todos los atributos de una partida, un valor por cada atributo. Cada valor puede ser de cualquier tipo, por ejemplo:
	\begin{itemize}
		\item Valor númerico.
		\item Una cadena de caracteres.
		\item Una letra.
		\item Una imagen.
		\item Un vídeo.
	\end{itemize}
	
	\item \textbf{RD1.4 Atributo} de una partida. Se encuentra añadido en la parte previa a la gestión de la base de datos.
	
	\item \textbf{RD1.5 Valor} de un atributo. Puede ser de cualquier tipo, por ejemplo:
	\begin{itemize}
		\item Valor númerico.
		\item Una cadena de caracteres.
		\item Una letra.
		\item Una imagen.
		\item Un vídeo.
	\end{itemize}
	
	\item \textbf{RD1.6 Atributos} de una partida. Es un conjunto de varios atributos añadidos en la parte previa a la gestión de la base de datos.
	
	\item \textbf{RD1.7 Valores de un mismo atributo}. Puede ser de cualquier tipo, aunque todos del mismo tipo, por ejemplo:
	\begin{itemize}
		\item Valor númerico.
		\item Una cadena de caracteres.
		\item Una letra.
		\item Una imagen.
		\item Un vídeo.
	\end{itemize}
	
	\item \textbf{RD1.8 Lista de condiciones}, de las cuales necesitan un valor de atributo, que son:
	\begin{itemize}
		\item 'sea menor que'.
		\item 'sea menor o igual que'.
		\item 'sea igual que'.
		\item 'sea mayor o igual que'.
		\item 'sea mayor que'.
		\item 'sea' (comparación entre cadena de caracteres).
		\item 'no sea' (comparación entre cadena de caracteres).
	\end{itemize}
	
	\item \textbf{RD1.9 Valores de varios atributos}. Pueden ser de cualquier tipo, por ejemplo:
	\begin{itemize}
		\item Valor númerico.
		\item Una cadena de caracteres.
		\item Una letra.
		\item Una imagen.
		\item Un vídeo.
	\end{itemize}
	
	\item \textbf{RD1.10 Juego}. Se compone de:
	\begin{itemize}
		\item Un identificador de juego.
	\end{itemize}
	
	\item \textbf{RD1.11 Juego}. Se compone de:
	\begin{itemize}
		\item Un identificador de juego.
	\end{itemize}
	
	\item \textbf{RD1.12 Atributos} de una partida. Es un conjunto de varios atributos añadidos en la parte previa a la gestión de la base de datos.
	
	\item \textbf{RD1.13 Lista de condiciones}, de las cuales necesitan un valor de atributo, que son:
	\begin{itemize}
		\item 'sea menor que'.
		\item 'sea menor o igual que'.
		\item 'sea igual que'.
		\item 'sea mayor o igual que'.
		\item 'sea mayor que'.
		\item 'sea' (comparación entre cadena de caracteres).
		\item 'no sea' (comparación entre cadena de caracteres).
	\end{itemize}
	
	\item \textbf{RD1.14 Valores de varios atributos}. Pueden ser de cualquier tipo, por ejemplo:
	\begin{itemize}
		\item Valor númerico.
		\item Una cadena de caracteres.
		\item Una letra.
		\item Una imagen.
		\item Un vídeo.
	\end{itemize}
	
	\item \textbf{RD1.15 Juego}. Se compone de:
	\begin{itemize}
		\item Un identificador de juego.
	\end{itemize}
	
	\item \textbf{RD1.16 Lista de condiciones}, de las cuales necesitan un valor de atributo, que son:
	\begin{itemize}
		\item 'sea menor que'.
		\item 'sea menor o igual que'.
		\item 'sea igual que'.
		\item 'sea mayor o igual que'.
		\item 'sea mayor que'.
		\item 'sea' (comparación entre cadena de caracteres).
		\item 'no sea' (comparación entre cadena de caracteres).
	\end{itemize}
	
	\item \textbf{RD1.17 Valores de varios atributos}. Pueden ser de cualquier tipo, por ejemplo:
	\begin{itemize}
		\item Valor númerico.
		\item Una cadena de caracteres.
		\item Una letra.
		\item Una imagen.
		\item Un vídeo.
	\end{itemize}
	
	\item \textbf{RD1.18 Valores de varios atributos}. Pueden ser de cualquier tipo, por ejemplo:
	\begin{itemize}
		\item Valor númerico.
		\item Una cadena de caracteres.
		\item Una letra.
		\item Una imagen.
		\item Un vídeo.
	\end{itemize}
	
	\item \textbf{RD1.19 Partidas} almacenadas de un juego concreto.
	
	\item \textbf{RD1.20 Partidas} almacenadas de un juego concreto.
	
	\item \textbf{RD1.21 Todos los valores de todos los atributos} almacenados de un juego concreto.
\end{itemize}




\subsubsection{Requisitos funcionales}
\begin{itemize}
	\item \textbf{RF1.1 Consultar partidas}: Consultar todas las partidas de un juego.
	
	Requisitos de entrada:
	\begin{itemize}
		\item RD1.1
	\end{itemize}
	Manejamiento de datos:
	\begin{itemize}
		\item RD1.19
	\end{itemize}
	Salida:
	\begin{itemize}
		\item RD1.3
	\end{itemize}
	
	
	\item \textbf{RF1.2 Restricción por condición}: Sobre un atributo, exigir que cumpla una condición dado un valor. Para facilitar esta función, sólo se puede exigir que se cumpla una condición sobre una lista de condiciones dada. Por ejemplo, Por ejemplo, si existiese un atributo en un juego que sólo tiene como valores {Ganado, Empatado, Perdido}, poder restringir a aquellas partidas donde dicho atributo sea 'Ganado'; o para el atributo 'Puntuación', poder elegir las partidas donde 'Se haya ganado por más de 10 puntos', donde la condición sobre el atributo 'Puntuación' sería 'es mayor que' y el valor pedido sería '10'.
	
	Requisitos de entrada: Se pasa el identificador del juego, el atributo a restringir, la lista de condiciones y el valor para la condición.
	\begin{itemize}
		\item RD1.10
		\item RD1.4
		\item RD1.8
		\item RD1.5
	\end{itemize}
	Manejamiento de datos: 
	\begin{itemize}
		\item RD1.21
	\end{itemize}
	Salida:
	\begin{itemize}
		\item RD1.7
	\end{itemize}
	
	
	\item \textbf{RF1.3 Unión de restricciones}: Poder hacer la función 'OR' entre restricciones en una consulta. Por ejemplo, 'todos los partidos en los que ganó o empató el Unicaja' o 'partidos en los que ganó el Unicaja o la diferencia en la puntuación fue menos de 10'.
	
	Requisitos de entrada: Se pasa el identificador del juego, los atributos a realizar la unión de restricciones, la lista de condiciones donde se selecciona una condición por atributo y los valores para cada condición.
	\begin{itemize}
		\item RD1.11
		\item RD1.6
		\item RD1.13
		\item RD1.9
	\end{itemize}
	Manejamiento de datos: 
	\begin{itemize}
		\item RD1.21
	\end{itemize}
	Salida:
	\begin{itemize}
		\item RD1.14
	\end{itemize}
	
	
	\item \textbf{RF1.4 Intersección de restricciones}: Poder hacer la función 'AND' entre restricciones en una consulta. Por ejemplo, 'todos los partidos en los que ganó y empató el Unicaja' o 'partidos en los que ganó el Unicaja y la diferencia en la puntuación fue menos de 10'.
	
	Requisitos de entrada: Se pasa el identificador del juego, los atributos a realizar la intersección de restricciones, la lista de condiciones donde se selecciona una condición por atributo y los valores para cada condición.
	\begin{itemize}
		\item RD1.15
		\item RD1.12
		\item RD1.16
		\item RD1.17
	\end{itemize}
	Manejamiento de datos: 
	\begin{itemize}
		\item RD1.21
	\end{itemize}
	Salida:
	\begin{itemize}
		\item RD1.18
	\end{itemize}
\end{itemize}




\subsubsection{Restricciones semánticas}

\textbf{RS1.1}. Para RF1.2, el valor de RD1.5 que acompaña a las concretas condiciones de 'es menor que', 'es menor o igual que', 'es igual que', 'es mayor o igual que', 'es mayor que' de la lista de condiciones RD1.8 ha de ser numérico.

\subsection{Segundo subsistema, consejos}
\susbsubsection{Requisitos de datos}

	\begin{itemize}
		\item \textbf{RD1: :}
	\end{itemize}


\subsubsection{Requisitos funcionales}

	\begin{itemize}
		\item \textbf{RF1: Recomendar equipos pokemon:}
		Requisitos de entrada:
		\begin{itemize}
			\item RD1 %% tabla tipos
		\end{itemize}

		Manejamiento de datos:
		\begin{itemize}
			\item RD2 %Estadisticas cada pkm
			\item RD3 %Equipos utilizados de un jugador
		\end{itemize}

		Salida:
		\begin{itemize}
			\item RD4 %Equipo pokemon
		\end{itemize}
	\end{itemize}


\subsubsection{Restricciones semánticas}


\subsection{Tercer subsistema, inclusión}
%TODO
\subsubsection{Requisito de datos}
\begin{itemize}
	\item \textbf{RD3.1: Atributos Principales}, proporcionados por el usuario, se componen de:
	\begin{itemize}
		\item Puntuación de la partida.
		\item Nombre del juego
	\end{itemize}
	
	\item \textbf{RD3.2: Atributos Secundarios}, se componen de:
	\begin{itemize}
		\item Nombre
		\item Valor del atributo
	\end{itemize}
	La lista de atributos secundarios la definirań los administradores del sistema.
\end{itemize}



\subsubsection{Requisitos funcionales}

\begin{itemize}
	\item \textbf{RF3.1 Selección del juego}: El usuario selecciona uno de entre los juegos que recoge el sistema.
	\begin{itemize}
		\item RD3.1
	\end{itemize}
	Salida: En pantalla se muestran los juegos posibles.
	
	\item \textbf{RF3.2 Inclusión de una partida}: El usuario inserta el atributo principal y crea un nuevo registro de partida.
	\begin{itemize}
		\item RD3.1
	\end{itemize}
	Requisitos de entrada: Una vez escogido el juego se inserta la puntuación de la partida.
	
	Manejo de datos: Se leen el atributo proporcionado y se el registro en la base de datos
	
	Salida: Mensaje de conformidad.
	
	
	\item \textbf{RF3.3 Modificación de un registro}: Una vez creado, el usuario podrá añadir atributos opcionales al registro.
	
	Requisitos de entrada: Lista de los atributos a insertar
	\begin{itemize}
		\item RF3.2
	\end{itemize}
	Manejo de datos: Se leen los atributos seleccionados y se inserta la gráfica en la base de datos.
	\begin{itemize}
		\item RD3.2
	\end{itemize}
	Salida: A pantalla se muestra una interfaz en la que se representa el registro completo.
	
\end{itemize}	
	
	\subsubsection{Restricciones semánticas}


\subsection{Cuarto subsistema, estadísticas}
%TODO
\subsubsection{Requisito de datos}
\begin{itemize}
	\item \textbf{RD4.1: Atributos de entrada a una gráfica 2D}, proporcionados por el usuario, se componen de:
	\begin{itemize}
		\item Atributo de la partida sobre la X
		\item Atributo de la partida sobre la Y
	\end{itemize}
	
	\item \textbf{RD4.2: Atributos de una partida}, se componen de:
	\begin{itemize}
		\item Nombre
		\item Valor del atributo
	\end{itemize}
	
\end{itemize}
\subsubsection{Requisitos funcionales}
\begin{itemize}
	\item \textbf{RF4.1 Realizar gráfica 2D de unos atributos}: El usuario selecciona dos atributos, de los cuales se realiza una gráfica 2D
	
	Requisitos de entrada: Escoger gráfica 2D y elegir qué dos atributos se escogen
	\begin{itemize}
		\item RD4.1
	\end{itemize}
	Manejo de datos: Se leen los dos atributos seleccionados  y se inserta la gráfica en la base de datos
	\begin{itemize}
		\item RD4.2
	\end{itemize}
	Salida: Gráfica 2D
	\begin{itemize}
		\item RD4.3
	\end{itemize}
	
	\item \textbf{RF4.2 Realizar gráfica 3D de unos atributos}: El usuario selecciona tres atributos, de los cuales se realiza una gráfica 3D
	
	Requisitos de entrada: Escoger gráfica 3D y elegir qué tres atributos se escogen
	\begin{itemize}
		\item RD4.4
	\end{itemize}
	Manejo de datos: Se leen los tres atributos seleccionados  y se inserta la gráfica en la base de datos
	\begin{itemize}
		\item RD4.5
	\end{itemize}
	Salida: Gráfica 3D
	\begin{itemize}
		\item RD4.6
	\end{itemize}
	
	\item \textbf{RF4.3 Realizar gráfica de columnas agrupadas de unos atributos}: El usuario selecciona dos atributos para las columnas y un tercero para el eje de abscisas, de los cuales se realiza la gráfica
	
	Requisitos de entrada: Escoger gráfica columnas agrupadas y elegir qué atributos se escogen
	\begin{itemize}
		\item RD4.7
	\end{itemize}
	Manejo de datos: Se leen los atributos seleccionados y se inserta la gráfica en la base de datos.
	\begin{itemize}
		\item RD4.8
	\end{itemize}
	Salida: Gráfica
	\begin{itemize}
		\item RD4.9
	\end{itemize}
	
	\item \textbf{RF4.4 Realizar gráfica circular de unos atributos}: El usuario selecciona un atributo, del cual se hace una gráfica porcentual en forma de círculo
	
	Requisitos de entrada: Escoger gráfica circular y elegir qué atributos se escoge.
	\begin{itemize}
		\item RD4.10
	\end{itemize}
	Manejo de datos: Se lee el atributo seleccionado y se inserta la gráfica en la base de datos.
	\begin{itemize}
		\item RD4.11
	\end{itemize}
	Salida: Gráfica.
	\begin{itemize}
		\item RD4.12
	\end{itemize}
	
	\item \textbf{RF4.5 Realizar una función elemental sobre un atributo concreto}: El usuario selecciona una función elemental (detalladas en la entrada de este requisito) y un atributo sobre el que operar, y se devuelve el resultado de la operación.
	
	Requisitos de entrada: El atributo y la función, a elegir entre media, varianza, suma, contar, máximo o mínimo.
	\begin{itemize}
		\item RD4.13
	\end{itemize}
	Manejo de datos: Se lee el atributo seleccionado y se opera.
	\begin{itemize}
		\item RD4.14
	\end{itemize}
	Salida: Resultado de la función.
	\begin{itemize}
		\item RD4.15
	\end{itemize}
	
	\item \textbf{RF4.6 Conseguir las últimas gráficas pedidas}: El usuario le da a un botón para ver las últimas gráficas que ha pedido.
	
	Requisitos de entrada: La activación de un botón.
	\begin{itemize}
		\item RD4.16
	\end{itemize}
	Manejo de datos: Se cogen las gráficas realizadas en función de su fecha.
	\begin{itemize}
		\item RD4.17
	\end{itemize}
	Salida: Gráficas, ordenadas de más a menos recientes.
	\begin{itemize}
		\item RD4.18
	\end{itemize}
	
\end{itemize}

\subsubsection{Restricciones semánticas}

\newpage

\subsection{Diagramas conjuntos: Armazón y Caja Negra}
\input{Diag/ArmazonYCajaNegra.tex}

\subsection{Diagramas del segundo subsistema, consejos}
\subsubsection{Diagramas de flujo de datos:}

Refinamos el proceso Consejos del diagrama armazón mediante
una primitiva descendente de descomposición en procesos sin conexiones.\\
Simultaneamente realizamos una descomposición de almacen sobre el
almacén MateDB del armazón.\\
Por último conectamos las descomposiciones resultantes mediante dos
primitivas ascendentes de generación de flujo.\\ 
 
\begin{figure}[H]
\centering
\includegraphics[width=0.7\linewidth]{../Diagramas/pdf/RefinamientoConsejos.pdf}
\caption[Flujo de datos del subsistema Consejos]{}
\caption{}
\label{fig:RefinamientoConsejos}
\end{figure}


\subsection{Diagramas del tercer subsistema, inclusión}

\begin{figure}[h!]
	\centering
	\includegraphics[width=0.7\linewidth]{../Diagramas/pdf/ER-Inclusion.pdf}
	\caption{Diagrama entidad relación del sistema de inclusión}
	\label{fig:ER}
\end{figure}



\begin{figure}[h!]
\centering
\includegraphics[width=0.7\linewidth]{../Diagramas/pdf/RefinamientoInclusion.pdf}
\caption{Refinamiento del sistema de inclusión}
\label{fig:RefinamientoInclusion}
\end{figure}


\end{document}
